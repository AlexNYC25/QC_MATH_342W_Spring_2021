\documentclass[12pt]{article}

\usepackage[margin=1.1in]{geometry}
%packages
%\usepackage{latexsym}
\usepackage{graphicx}
\usepackage{color}
\usepackage{amsmath}
\usepackage{dsfont}
\usepackage{placeins}
\usepackage{amssymb}
\usepackage{wasysym}
\usepackage{abstract}
\usepackage{hyperref}
\usepackage{etoolbox}
\usepackage{datetime}
\usepackage{xcolor}
\usepackage{alphalph}
\settimeformat{ampmtime}

%\usepackage{pstricks,pst-node,pst-tree}

%\usepackage{algpseudocode}
%\usepackage{amsthm}
%\usepackage{hyperref}
%\usepackage{mathrsfs}
%\usepackage{amsfonts}
%\usepackage{bbding}
%\usepackage{listings}
%\usepackage{appendix}
\usepackage[margin=1in]{geometry}
%\geometry{papersize={8.5in,11in},total={6.5in,9in}}
%\usepackage{cancel}
%\usepackage{algorithmic, algorithm}

\makeatletter
\def\maxwidth{ %
  \ifdim\Gin@nat@width>\linewidth
    \linewidth
  \else
    \Gin@nat@width
  \fi
}
\makeatother

\definecolor{fgcolor}{rgb}{0.345, 0.345, 0.345}
\newcommand{\hlnum}[1]{\textcolor[rgb]{0.686,0.059,0.569}{#1}}%
\newcommand{\hlstr}[1]{\textcolor[rgb]{0.192,0.494,0.8}{#1}}%
\newcommand{\hlcom}[1]{\textcolor[rgb]{0.678,0.584,0.686}{\textit{#1}}}%
\newcommand{\hlopt}[1]{\textcolor[rgb]{0,0,0}{#1}}%
\newcommand{\hlstd}[1]{\textcolor[rgb]{0.345,0.345,0.345}{#1}}%
\newcommand{\hlkwa}[1]{\textcolor[rgb]{0.161,0.373,0.58}{\textbf{#1}}}%
\newcommand{\hlkwb}[1]{\textcolor[rgb]{0.69,0.353,0.396}{#1}}%
\newcommand{\hlkwc}[1]{\textcolor[rgb]{0.333,0.667,0.333}{#1}}%
\newcommand{\hlkwd}[1]{\textcolor[rgb]{0.737,0.353,0.396}{\textbf{#1}}}%

\usepackage{framed}
\makeatletter
\newenvironment{kframe}{%
 \def\at@end@of@kframe{}%
 \ifinner\ifhmode%
  \def\at@end@of@kframe{\end{minipage}}%
  \begin{minipage}{\columnwidth}%
 \fi\fi%
 \def\FrameCommand##1{\hskip\@totalleftmargin \hskip-\fboxsep
 \colorbox{shadecolor}{##1}\hskip-\fboxsep
     % There is no \\@totalrightmargin, so:
     \hskip-\linewidth \hskip-\@totalleftmargin \hskip\columnwidth}%
 \MakeFramed {\advance\hsize-\width
   \@totalleftmargin\z@ \linewidth\hsize
   \@setminipage}}%
 {\par\unskip\endMakeFramed%
 \at@end@of@kframe}
\makeatother

\definecolor{shadecolor}{rgb}{.77, .77, .77}
\definecolor{messagecolor}{rgb}{0, 0, 0}
\definecolor{warningcolor}{rgb}{1, 0, 1}
\definecolor{errorcolor}{rgb}{1, 0, 0}
\newenvironment{knitrout}{}{} % an empty environment to be redefined in TeX

\usepackage{alltt}
\usepackage[T1]{fontenc}

\newcommand{\qu}[1]{``#1''}
\newcounter{probnum}
\setcounter{probnum}{1}

%create definition to allow local margin changes
\def\changemargin#1#2{\list{}{\rightmargin#2\leftmargin#1}\item[]}
\let\endchangemargin=\endlist 

%allow equations to span multiple pages
\allowdisplaybreaks

%define colors and color typesetting conveniences
\definecolor{gray}{rgb}{0.5,0.5,0.5}
\definecolor{black}{rgb}{0,0,0}
\definecolor{white}{rgb}{1,1,1}
\definecolor{blue}{rgb}{0.5,0.5,1}
\newcommand{\inblue}[1]{\color{blue}#1 \color{black}}
\definecolor{green}{rgb}{0.133,0.545,0.133}
\newcommand{\ingreen}[1]{\color{green}#1 \color{black}}
\definecolor{yellow}{rgb}{1,1,0}
\newcommand{\inyellow}[1]{\color{yellow}#1 \color{black}}
\definecolor{orange}{rgb}{0.9,0.649,0}
\newcommand{\inorange}[1]{\color{orange}#1 \color{black}}
\definecolor{red}{rgb}{1,0.133,0.133}
\newcommand{\inred}[1]{\color{red}#1 \color{black}}
\definecolor{purple}{rgb}{0.58,0,0.827}
\newcommand{\inpurple}[1]{\color{purple}#1 \color{black}}
\definecolor{backgcode}{rgb}{0.97,0.97,0.8}
\definecolor{Brown}{cmyk}{0,0.81,1,0.60}
\definecolor{OliveGreen}{cmyk}{0.64,0,0.95,0.40}
\definecolor{CadetBlue}{cmyk}{0.62,0.57,0.23,0}

%define new math operators
\DeclareMathOperator*{\argmax}{arg\,max~}
\DeclareMathOperator*{\argmin}{arg\,min~}
\DeclareMathOperator*{\argsup}{arg\,sup~}
\DeclareMathOperator*{\arginf}{arg\,inf~}
\DeclareMathOperator*{\convolution}{\text{\Huge{$\ast$}}}
\newcommand{\infconv}[2]{\convolution^\infty_{#1 = 1} #2}
%true functions

%%%% GENERAL SHORTCUTS

%shortcuts for pure typesetting conveniences
\newcommand{\bv}[1]{\boldsymbol{#1}}

%shortcuts for compound constants
\newcommand{\BetaDistrConst}{\dfrac{\Gamma(\alpha + \beta)}{\Gamma(\alpha)\Gamma(\beta)}}
\newcommand{\NormDistrConst}{\dfrac{1}{\sqrt{2\pi\sigma^2}}}

%shortcuts for conventional symbols
\newcommand{\tsq}{\tau^2}
\newcommand{\tsqh}{\hat{\tau}^2}
\newcommand{\sigsq}{\sigma^2}
\newcommand{\sigsqsq}{\parens{\sigma^2}^2}
\newcommand{\sigsqovern}{\dfrac{\sigsq}{n}}
\newcommand{\tausq}{\tau^2}
\newcommand{\tausqalpha}{\tau^2_\alpha}
\newcommand{\tausqbeta}{\tau^2_\beta}
\newcommand{\tausqsigma}{\tau^2_\sigma}
\newcommand{\betasq}{\beta^2}
\newcommand{\sigsqvec}{\bv{\sigma}^2}
\newcommand{\sigsqhat}{\hat{\sigma}^2}
\newcommand{\sigsqhatmlebayes}{\sigsqhat_{\text{Bayes, MLE}}}
\newcommand{\sigsqhatmle}[1]{\sigsqhat_{#1, \text{MLE}}}
\newcommand{\bSigma}{\bv{\Sigma}}
\newcommand{\bSigmainv}{\bSigma^{-1}}
\newcommand{\thetavec}{\bv{\theta}}
\newcommand{\thetahat}{\hat{\theta}}
\newcommand{\thetahatmle}{\hat{\theta}_{\mathrm{MLE}}}
\newcommand{\thetavechatmle}{\hat{\thetavec}_{\mathrm{MLE}}}
\newcommand{\muhat}{\hat{\mu}}
\newcommand{\musq}{\mu^2}
\newcommand{\muvec}{\bv{\mu}}
\newcommand{\muhatmle}{\muhat_{\text{MLE}}}
\newcommand{\lambdahat}{\hat{\lambda}}
\newcommand{\lambdahatmle}{\lambdahat_{\text{MLE}}}
\newcommand{\etavec}{\bv{\eta}}
\newcommand{\alphavec}{\bv{\alpha}}
\newcommand{\minimaxdec}{\delta^*_{\mathrm{mm}}}
\newcommand{\ybar}{\bar{y}}
\newcommand{\xbar}{\bar{x}}
\newcommand{\Xbar}{\bar{X}}
\newcommand{\phat}{\hat{p}}
\newcommand{\Phat}{\hat{P}}
\newcommand{\Zbar}{\bar{Z}}
\newcommand{\iid}{~{\buildrel iid \over \sim}~}
\newcommand{\inddist}{~{\buildrel ind \over \sim}~}
\newcommand{\approxdist}{~{\buildrel approx \over \sim}~}
\newcommand{\equalsindist}{~{\buildrel d \over =}~}
\newcommand{\loglik}[1]{\ell\parens{#1}}
\newcommand{\thetahatkminone}{\thetahat^{(k-1)}}
\newcommand{\thetahatkplusone}{\thetahat^{(k+1)}}
\newcommand{\thetahatk}{\thetahat^{(k)}}
\newcommand{\half}{\frac{1}{2}}
\newcommand{\third}{\frac{1}{3}}
\newcommand{\twothirds}{\frac{2}{3}}
\newcommand{\fourth}{\frac{1}{4}}
\newcommand{\fifth}{\frac{1}{5}}
\newcommand{\sixth}{\frac{1}{6}}

%shortcuts for vector and matrix notation
\newcommand{\A}{\bv{A}}
\newcommand{\At}{\A^T}
\newcommand{\Ainv}{\inverse{\A}}
\newcommand{\B}{\bv{B}}
\newcommand{\K}{\bv{K}}
\newcommand{\Kt}{\K^T}
\newcommand{\Kinv}{\inverse{K}}
\newcommand{\Kinvt}{(\Kinv)^T}
\newcommand{\M}{\bv{M}}
\newcommand{\Bt}{\B^T}
\newcommand{\Q}{\bv{Q}}
\newcommand{\Qt}{\Q^T}
\newcommand{\R}{\bv{R}}
\newcommand{\Rt}{\R^T}
\newcommand{\Z}{\bv{Z}}
\newcommand{\X}{\bv{X}}
\newcommand{\Xsub}{\X_{\text{(sub)}}}
\newcommand{\Xsubadj}{\X_{\text{(sub,adj)}}}
\newcommand{\I}{\bv{I}}
\newcommand{\Y}{\bv{Y}}
\newcommand{\sigsqI}{\sigsq\I}
\renewcommand{\P}{\bv{P}}
\newcommand{\Psub}{\P_{\text{(sub)}}}
\newcommand{\Pt}{\P^T}
\newcommand{\Pii}{P_{ii}}
\newcommand{\Pij}{P_{ij}}
\newcommand{\IminP}{(\I-\P)}
\newcommand{\Xt}{\bv{X}^T}
\newcommand{\XtX}{\Xt\X}
\newcommand{\XtXinv}{\parens{\Xt\X}^{-1}}
\newcommand{\XtXinvXt}{\XtXinv\Xt}
\newcommand{\XXtXinvXt}{\X\XtXinvXt}
\newcommand{\x}{\bv{x}}
\newcommand{\onevec}{\bv{1}}
\newcommand{\oneton}{1, \ldots, n}
\newcommand{\yoneton}{y_1, \ldots, y_n}
\newcommand{\yonetonorder}{y_{(1)}, \ldots, y_{(n)}}
\newcommand{\Yoneton}{Y_1, \ldots, Y_n}
\newcommand{\iinoneton}{i \in \braces{\oneton}}
\newcommand{\onetom}{1, \ldots, m}
\newcommand{\jinonetom}{j \in \braces{\onetom}}
\newcommand{\xoneton}{x_1, \ldots, x_n}
\newcommand{\Xoneton}{X_1, \ldots, X_n}
\newcommand{\xt}{\x^T}
\newcommand{\y}{\bv{y}}
\newcommand{\yt}{\y^T}
\renewcommand{\c}{\bv{c}}
\newcommand{\ct}{\c^T}
\newcommand{\tstar}{\bv{t}^*}
\renewcommand{\u}{\bv{u}}
\renewcommand{\v}{\bv{v}}
\renewcommand{\a}{\bv{a}}
\newcommand{\s}{\bv{s}}
\newcommand{\yadj}{\y_{\text{(adj)}}}
\newcommand{\xjadj}{\x_{j\text{(adj)}}}
\newcommand{\xjadjM}{\x_{j \perp M}}
\newcommand{\yhat}{\hat{\y}}
\newcommand{\yhatsub}{\yhat_{\text{(sub)}}}
\newcommand{\yhatstar}{\yhat^*}
\newcommand{\yhatstarnew}{\yhatstar_{\text{new}}}
\newcommand{\z}{\bv{z}}
\newcommand{\zt}{\z^T}
\newcommand{\bb}{\bv{b}}
\newcommand{\bbt}{\bb^T}
\newcommand{\bbeta}{\bv{\beta}}
\newcommand{\beps}{\bv{\epsilon}}
\newcommand{\bepst}{\beps^T}
\newcommand{\e}{\bv{e}}
\newcommand{\Mofy}{\M(\y)}
\newcommand{\KofAlpha}{K(\alpha)}
\newcommand{\ellset}{\mathcal{L}}
\newcommand{\oneminalph}{1-\alpha}
\newcommand{\SSE}{\text{SSE}}
\newcommand{\SSEsub}{\text{SSE}_{\text{(sub)}}}
\newcommand{\MSE}{\text{MSE}}
\newcommand{\RMSE}{\text{RMSE}}
\newcommand{\SSR}{\text{SSR}}
\newcommand{\SST}{\text{SST}}
\newcommand{\JSest}{\delta_{\text{JS}}(\x)}
\newcommand{\Bayesest}{\delta_{\text{Bayes}}(\x)}
\newcommand{\EmpBayesest}{\delta_{\text{EmpBayes}}(\x)}
\newcommand{\BLUPest}{\delta_{\text{BLUP}}}
\newcommand{\MLEest}[1]{\hat{#1}_{\text{MLE}}}

%shortcuts for Linear Algebra stuff (i.e. vectors and matrices)
\newcommand{\twovec}[2]{\bracks{\begin{array}{c} #1 \\ #2 \end{array}}}
\newcommand{\threevec}[3]{\bracks{\begin{array}{c} #1 \\ #2 \\ #3 \end{array}}}
\newcommand{\fivevec}[5]{\bracks{\begin{array}{c} #1 \\ #2 \\ #3 \\ #4 \\ #5 \end{array}}}
\newcommand{\twobytwomat}[4]{\bracks{\begin{array}{cc} #1 & #2 \\ #3 & #4 \end{array}}}
\newcommand{\threebytwomat}[6]{\bracks{\begin{array}{cc} #1 & #2 \\ #3 & #4 \\ #5 & #6 \end{array}}}

%shortcuts for conventional compound symbols
\newcommand{\thetainthetas}{\theta \in \Theta}
\newcommand{\reals}{\mathbb{R}}
\newcommand{\complexes}{\mathbb{C}}
\newcommand{\rationals}{\mathbb{Q}}
\newcommand{\integers}{\mathbb{Z}}
\newcommand{\naturals}{\mathbb{N}}
\newcommand{\forallninN}{~~\forall n \in \naturals}
\newcommand{\forallxinN}[1]{~~\forall #1 \in \reals}
\newcommand{\matrixdims}[2]{\in \reals^{\,#1 \times #2}}
\newcommand{\inRn}[1]{\in \reals^{\,#1}}
\newcommand{\mathimplies}{\quad\Rightarrow\quad}
\newcommand{\mathlogicequiv}{\quad\Leftrightarrow\quad}
\newcommand{\eqncomment}[1]{\quad \text{(#1)}}
\newcommand{\limitn}{\lim_{n \rightarrow \infty}}
\newcommand{\limitN}{\lim_{N \rightarrow \infty}}
\newcommand{\limitd}{\lim_{d \rightarrow \infty}}
\newcommand{\limitt}{\lim_{t \rightarrow \infty}}
\newcommand{\limitsupn}{\limsup_{n \rightarrow \infty}~}
\newcommand{\limitinfn}{\liminf_{n \rightarrow \infty}~}
\newcommand{\limitk}{\lim_{k \rightarrow \infty}}
\newcommand{\limsupn}{\limsup_{n \rightarrow \infty}}
\newcommand{\limsupk}{\limsup_{k \rightarrow \infty}}
\newcommand{\floor}[1]{\left\lfloor #1 \right\rfloor}
\newcommand{\ceil}[1]{\left\lceil #1 \right\rceil}

%shortcuts for environments
\newcommand{\beqn}{\vspace{-0.25cm}\begin{eqnarray*}}
\newcommand{\eeqn}{\end{eqnarray*}}
\newcommand{\bneqn}{\vspace{-0.25cm}\begin{eqnarray}}
\newcommand{\eneqn}{\end{eqnarray}}

%shortcuts for mini environments
\newcommand{\parens}[1]{\left(#1\right)}
\newcommand{\squared}[1]{\parens{#1}^2}
\newcommand{\tothepow}[2]{\parens{#1}^{#2}}
\newcommand{\prob}[1]{\mathbb{P}\parens{#1}}
\newcommand{\cprob}[2]{\prob{#1~|~#2}}
\newcommand{\littleo}[1]{o\parens{#1}}
\newcommand{\bigo}[1]{O\parens{#1}}
\newcommand{\Lp}[1]{\mathbb{L}^{#1}}
\renewcommand{\arcsin}[1]{\text{arcsin}\parens{#1}}
\newcommand{\prodonen}[2]{\bracks{\prod_{#1=1}^n #2}}
\newcommand{\mysum}[4]{\sum_{#1=#2}^{#3} #4}
\newcommand{\sumonen}[2]{\sum_{#1=1}^n #2}
\newcommand{\infsum}[2]{\sum_{#1=1}^\infty #2}
\newcommand{\infprod}[2]{\prod_{#1=1}^\infty #2}
\newcommand{\infunion}[2]{\bigcup_{#1=1}^\infty #2}
\newcommand{\infinter}[2]{\bigcap_{#1=1}^\infty #2}
\newcommand{\infintegral}[2]{\int^\infty_{-\infty} #2 ~\text{d}#1}
\newcommand{\supthetas}[1]{\sup_{\thetainthetas}\braces{#1}}
\newcommand{\bracks}[1]{\left[#1\right]}
\newcommand{\braces}[1]{\left\{#1\right\}}
\newcommand{\set}[1]{\left\{#1\right\}}
\newcommand{\abss}[1]{\left|#1\right|}
\newcommand{\norm}[1]{\left|\left|#1\right|\right|}
\newcommand{\normsq}[1]{\norm{#1}^2}
\newcommand{\inverse}[1]{\parens{#1}^{-1}}
\newcommand{\rowof}[2]{\parens{#1}_{#2\cdot}}

%shortcuts for functionals
\newcommand{\realcomp}[1]{\text{Re}\bracks{#1}}
\newcommand{\imagcomp}[1]{\text{Im}\bracks{#1}}
\newcommand{\range}[1]{\text{range}\bracks{#1}}
\newcommand{\colsp}[1]{\text{colsp}\bracks{#1}}
\newcommand{\rowsp}[1]{\text{rowsp}\bracks{#1}}
\newcommand{\tr}[1]{\text{tr}\bracks{#1}}
\newcommand{\rank}[1]{\text{rank}\bracks{#1}}
\newcommand{\proj}[2]{\text{Proj}_{#1}\bracks{#2}}
\newcommand{\projcolspX}[1]{\text{Proj}_{\colsp{\X}}\bracks{#1}}
\newcommand{\median}[1]{\text{median}\bracks{#1}}
\newcommand{\mean}[1]{\text{mean}\bracks{#1}}
\newcommand{\dime}[1]{\text{dim}\bracks{#1}}
\renewcommand{\det}[1]{\text{det}\bracks{#1}}
\newcommand{\expe}[1]{\mathbb{E}\bracks{#1}}
\newcommand{\expeabs}[1]{\expe{\abss{#1}}}
\newcommand{\expesub}[2]{\mathbb{E}_{#1}\bracks{#2}}
\newcommand{\indic}[1]{\mathds{1}_{#1}}
\newcommand{\var}[1]{\mathbb{V}\text{ar}\bracks{#1}}
\newcommand{\cov}[2]{\mathbb{C}\text{ov}\bracks{#1, #2}}
\newcommand{\corr}[2]{\text{Corr}\bracks{#1, #2}}
\newcommand{\se}[1]{\mathbb{S}\text{E}\bracks{#1}}
\newcommand{\seest}[1]{\hat{\mathbb{S}\text{E}}\bracks{#1}}
\newcommand{\bias}[1]{\text{Bias}\bracks{#1}}
\newcommand{\derivop}[2]{\dfrac{\text{d}}{\text{d} #1}\bracks{#2}}
\newcommand{\partialop}[2]{\dfrac{\partial}{\partial #1}\bracks{#2}}
\newcommand{\secpartialop}[2]{\dfrac{\partial^2}{\partial #1^2}\bracks{#2}}
\newcommand{\mixpartialop}[3]{\dfrac{\partial^2}{\partial #1 \partial #2}\bracks{#3}}

%shortcuts for functions
\renewcommand{\exp}[1]{\mathrm{exp}\parens{#1}}
\renewcommand{\cos}[1]{\text{cos}\parens{#1}}
\renewcommand{\sin}[1]{\text{sin}\parens{#1}}
\newcommand{\sign}[1]{\text{sign}\parens{#1}}
\newcommand{\are}[1]{\mathrm{ARE}\parens{#1}}
\newcommand{\natlog}[1]{\ln\parens{#1}}
\newcommand{\oneover}[1]{\frac{1}{#1}}
\newcommand{\overtwo}[1]{\frac{#1}{2}}
\newcommand{\overn}[1]{\frac{#1}{n}}
\newcommand{\oneoversqrt}[1]{\oneover{\sqrt{#1}}}
\newcommand{\sqd}[1]{\parens{#1}^2}
\newcommand{\loss}[1]{\ell\parens{\theta, #1}}
\newcommand{\losstwo}[2]{\ell\parens{#1, #2}}
\newcommand{\cf}{\phi(t)}

%English language specific shortcuts
\newcommand{\ie}{\textit{i.e.} }
\newcommand{\AKA}{\textit{AKA} }
\renewcommand{\iff}{\textit{iff}}
\newcommand{\eg}{\textit{e.g.} }
\newcommand{\st}{\textit{s.t.} }
\newcommand{\wrt}{\textit{w.r.t.} }
\newcommand{\mathst}{~~\text{\st}~~}
\newcommand{\mathand}{~~\text{and}~~}
\newcommand{\ala}{\textit{a la} }
\newcommand{\ppp}{posterior predictive p-value}
\newcommand{\dd}{dataset-to-dataset}

%shortcuts for distribution titles
\newcommand{\logistic}[2]{\mathrm{Logistic}\parens{#1,\,#2}}
\newcommand{\bernoulli}[1]{\mathrm{Bernoulli}\parens{#1}}
\newcommand{\betanot}[2]{\mathrm{Beta}\parens{#1,\,#2}}
\newcommand{\stdbetanot}{\betanot{\alpha}{\beta}}
\newcommand{\multnormnot}[3]{\mathcal{N}_{#1}\parens{#2,\,#3}}
\newcommand{\normnot}[2]{\mathcal{N}\parens{#1,\,#2}}
\newcommand{\classicnormnot}{\normnot{\mu}{\sigsq}}
\newcommand{\stdnormnot}{\normnot{0}{1}}
\newcommand{\uniformdiscrete}[1]{\mathrm{Uniform}\parens{\braces{#1}}}
\newcommand{\uniform}[2]{\mathrm{U}\parens{#1,\,#2}}
\newcommand{\stduniform}{\uniform{0}{1}}
\newcommand{\geometric}[1]{\mathrm{Geometric}\parens{#1}}
\newcommand{\hypergeometric}[3]{\mathrm{Hypergeometric}\parens{#1,\,#2,\,#3}}
\newcommand{\exponential}[1]{\mathrm{Exp}\parens{#1}}
\newcommand{\gammadist}[2]{\mathrm{Gamma}\parens{#1, #2}}
\newcommand{\poisson}[1]{\mathrm{Poisson}\parens{#1}}
\newcommand{\binomial}[2]{\mathrm{Binomial}\parens{#1,\,#2}}
\newcommand{\negbin}[2]{\mathrm{NegBin}\parens{#1,\,#2}}
\newcommand{\rayleigh}[1]{\mathrm{Rayleigh}\parens{#1}}
\newcommand{\multinomial}[2]{\mathrm{Multinomial}\parens{#1,\,#2}}
\newcommand{\gammanot}[2]{\mathrm{Gamma}\parens{#1,\,#2}}
\newcommand{\cauchynot}[2]{\text{Cauchy}\parens{#1,\,#2}}
\newcommand{\invchisqnot}[1]{\text{Inv}\chisq{#1}}
\newcommand{\invscaledchisqnot}[2]{\text{ScaledInv}\ncchisq{#1}{#2}}
\newcommand{\invgammanot}[2]{\text{InvGamma}\parens{#1,\,#2}}
\newcommand{\chisq}[1]{\chi^2_{#1}}
\newcommand{\ncchisq}[2]{\chi^2_{#1}\parens{#2}}
\newcommand{\ncF}[3]{F_{#1,#2}\parens{#3}}

%shortcuts for PDF's of common distributions
\newcommand{\logisticpdf}[3]{\oneover{#3}\dfrac{\exp{-\dfrac{#1 - #2}{#3}}}{\parens{1+\exp{-\dfrac{#1 - #2}{#3}}}^2}}
\newcommand{\betapdf}[3]{\dfrac{\Gamma(#2 + #3)}{\Gamma(#2)\Gamma(#3)}#1^{#2-1} (1-#1)^{#3-1}}
\newcommand{\normpdf}[3]{\frac{1}{\sqrt{2\pi#3}}\exp{-\frac{1}{2#3}(#1 - #2)^2}}
\newcommand{\normpdfvarone}[2]{\dfrac{1}{\sqrt{2\pi}}e^{-\half(#1 - #2)^2}}
\newcommand{\chisqpdf}[2]{\dfrac{1}{2^{#2/2}\Gamma(#2/2)}\; {#1}^{#2/2-1} e^{-#1/2}}
\newcommand{\invchisqpdf}[2]{\dfrac{2^{-\overtwo{#1}}}{\Gamma(#2/2)}\,{#1}^{-\overtwo{#2}-1}  e^{-\oneover{2 #1}}}
\newcommand{\exponentialpdf}[2]{#2\exp{-#2#1}}
\newcommand{\poissonpdf}[2]{\dfrac{e^{-#1} #1^{#2}}{#2!}}
\newcommand{\binomialpdf}[3]{\binom{#2}{#1}#3^{#1}(1-#3)^{#2-#1}}
\newcommand{\rayleighpdf}[2]{\dfrac{#1}{#2^2}\exp{-\dfrac{#1^2}{2 #2^2}}}
\newcommand{\gammapdf}[3]{\dfrac{#3^#2}{\Gamma\parens{#2}}#1^{#2-1}\exp{-#3 #1}}
\newcommand{\cauchypdf}[3]{\oneover{\pi} \dfrac{#3}{\parens{#1-#2}^2 + #3^2}}
\newcommand{\Gammaf}[1]{\Gamma\parens{#1}}

%shortcuts for miscellaneous typesetting conveniences
\newcommand{\notesref}[1]{\marginpar{\color{gray}\tt #1\color{black}}}

%%%% DOMAIN-SPECIFIC SHORTCUTS

%Real analysis related shortcuts
\newcommand{\zeroonecl}{\bracks{0,1}}
\newcommand{\forallepsgrzero}{\forall \epsilon > 0~~}
\newcommand{\lessthaneps}{< \epsilon}
\newcommand{\fraccomp}[1]{\text{frac}\bracks{#1}}

%Bayesian related shortcuts
\newcommand{\yrep}{y^{\text{rep}}}
\newcommand{\yrepisq}{(\yrep_i)^2}
\newcommand{\yrepvec}{\bv{y}^{\text{rep}}}


%Probability shortcuts
\newcommand{\SigField}{\mathcal{F}}
\newcommand{\ProbMap}{\mathcal{P}}
\newcommand{\probtrinity}{\parens{\Omega, \SigField, \ProbMap}}
\newcommand{\convp}{~{\buildrel p \over \rightarrow}~}
\newcommand{\convLp}[1]{~{\buildrel \Lp{#1} \over \rightarrow}~}
\newcommand{\nconvp}{~{\buildrel p \over \nrightarrow}~}
\newcommand{\convae}{~{\buildrel a.e. \over \longrightarrow}~}
\newcommand{\convau}{~{\buildrel a.u. \over \longrightarrow}~}
\newcommand{\nconvau}{~{\buildrel a.u. \over \nrightarrow}~}
\newcommand{\nconvae}{~{\buildrel a.e. \over \nrightarrow}~}
\newcommand{\convd}{~{\buildrel \mathcal{D} \over \rightarrow}~}
\newcommand{\nconvd}{~{\buildrel \mathcal{D} \over \nrightarrow}~}
\newcommand{\withprob}{~~\text{w.p.}~~}
\newcommand{\io}{~~\text{i.o.}}

\newcommand{\Acl}{\bar{A}}
\newcommand{\ENcl}{\bar{E}_N}
\newcommand{\diam}[1]{\text{diam}\parens{#1}}

\newcommand{\taua}{\tau_a}

\newcommand{\myint}[4]{\int_{#2}^{#3} #4 \,\text{d}#1}
\newcommand{\laplacet}[1]{\mathscr{L}\bracks{#1}}
\newcommand{\laplaceinvt}[1]{\mathscr{L}^{-1}\bracks{#1}}
\renewcommand{\min}[1]{\text{min}\braces{#1}}
\renewcommand{\max}[1]{\text{max}\braces{#1}}

\newcommand{\Vbar}[1]{\bar{V}\parens{#1}}
\newcommand{\expnegrtau}{\exp{-r\tau}}

%%% problem typesetting
\definecolor{darkgrey}{rgb}{0.10,0.10,0.9}

\newcommand{\problem}[1]{\noindent \colorbox{black}{{\color{yellow} \large{\textsf{\textbf{Problem \arabic{probnum}}}}~}} \addtocounter{probnum}{1} \vspace{0.2cm} \\ \iftoggle{professormode}{}{\color{darkgrey}} #1}

\newcommand{\easysubproblem}[1]{\ingreen{\item} \iftoggle{professormode}{}{\color{darkgrey}} [easy] #1 \color{black} }
\newcommand{\intermediatesubproblem}[1]{\inorange{\item} \iftoggle{professormode}{}{\color{darkgrey}} [harder] #1 \color{black} }
\newcommand{\hardsubproblem}[1]{\inred{\item} \iftoggle{professormode}{}{\color{darkgrey}} [difficult] #1 \color{black} }
\newcommand{\extracreditsubproblem}[1]{\inpurple{\item} \iftoggle{professormode}{}{\color{darkgrey}} [E.C.] #1 \color{black} }


\newcommand{\spc}[1]{\iftoggle{professormode}{\\ \vspace{#1cm}}{\\ \vspace{-0.3cm}}}

\makeatletter
\newalphalph{\alphmult}[mult]{\@alph}{26}
\renewcommand{\labelenumi}{(\alphmult{\value{enumi}})}

\newcommand{\support}[1]{\text{Supp}\bracks{#1}}
\newcommand{\mode}[1]{\text{Mode}\bracks{#1}}
\newcommand{\IQR}[1]{\text{IQR}\bracks{#1}}
\newcommand{\quantile}[2]{\text{Quantile}\bracks{#1,\,#2}}


\newcommand{\coursedept}{Math}
\newcommand{\coursenumber}{342W}
\newcommand{\coursenumbercrosslisted}{/ 650.03~}
\newcommand{\semester}{Spring}
\newcommand{\numcredits}{6}
\newcommand{\lectimeandloc}{Mon and Wed 5-6:50PM / on zoom}
\newcommand{\requiredlabtimeandloc}{Required Lab Time / Loc 		& Thurs 9-10:50AM / on zoom \\}
\newcommand{\tataofficehourtimeandloc}{TA 	/ TA Office Hours / Loc 			& Tzipora Horowitz / Wed 6:55-7:55PM / on zoom}
\newcommand{\coursewebpageurl}{https://github.com/kapelner/QC_\coursedept_\coursenumber_\semester_\the\year}
\newcommand{\coursewebpagelink}{\href{\coursewebpageurl}{course homepage}}
\newcommand{\slackurl}{https://QC\coursedept\coursenumber\semester\the\year.slack.com/}
\newcommand{\slacklink}{\href{\slackurl}{slack}}
\newcommand{\numtheoryhws}{4--7}
\newcommand{\extrahwzero}{\item provide a link to your public repository on github (this means you need to sign up for github first)}
\newcommand{\hwzerodue}{Wednesday, Feb 3 11:59PM}
\newcommand{\lastdatetimetohandinhomeworks}{May 18 at noon}

\input{../../syllabi/_header}

\section*{Course Overview}

MATH 342W. Data Science via Machine Learning and Statistical Modeling. 6 hr.; 4 cr. Prereq.: MATH 241 (intro to probablity and statistics), MATH 231 (intro to linear algebra), CSCI 111 (intro to programming) or equivalents. Recommended: ECON 382 (intro to economentrics) or equivalent. Philosophy of modeling and learning using data. Prediction via the ordinary linear model including orthogonal projections, sum of squares identity, $R^2$ and RMSE. Polynomial and interaction regressions. Prediction with machine learning including neural nets (the perceptron), support vector machines and the tree methods CART, bagged trees and Random Forests. Probability estimation using logistic regression, asymmetric cost classifiers and the ROC / DET performance curves. Underfitting vs. overfitting and the bias-variance decomposition / tradeoff. Model validation including out of sample techniques such as cross validation and bootstrap validation. Correlation vs. causation, causal models, lurking variables and interpretations of linear model coefficients. Extrapolation. The \texttt{R} language will be taught formally from the ground and up (its use will be a substantial part of the homework) as well as visualization using the \texttt{ggplot} library and manipulation using the \texttt{dplyr} and \texttt{data.table} libraries. \pagebreak

You should be familiar with the following before entering the class:

\begin{itemize}
\itemsep -0.0em 
\item Basic Probability Theory: conditional probability, in/dependence, identical distributedness
\item Modeling with discrete random variables: Bernoulli, Binomial%, Poisson, Geometric, Negative Binomial, Uniform Discrete and others
\item Expectation and variance
%\item Modeling with continuous random variables: Exponential, Uniform and Normal
%\item Frequentist confidence intervals and hypothesis testing for one-sample proportions
%\item Basic visualization of data: plots, histograms, bar charts
\item Linear algebra: Vectors, matrices, rank, transpose
\item Programming: basic data types, vectors, arrays, control flow (for, while, if, else), functions
\end{itemize}

\noindent We will review the above \textit{throughout the semester} when needed and we will do so rapidly. \\

\textbf{This is not your typical mathematics course.} This course will do lots of modeling of real-world situations using data via the \texttt{R} statistical language.



\section*{Course Materials}

We will be using many reference texts and three popular books which you will read portions from. However the main materials are the course notes. You should always supplement concepts from class by reading up on them online; \href{https://en.wikipedia.org}{wikipedia} I find the best for this. 

\paragraph{Theory Reference:} It is not necessary to have these two books, but it is recommended. The first is \qu{Learning from Data: A Short Course} by Abu-Mostafa, Magdon-Ismael and Lin which can be purchased used on \href{https://www.amazon.com/Learning-Data-Yaser-S-Abu-Mostafa/dp/1600490069}{Amazon}. We will also be using portions from \qu{Deep Learning} by Goodfellow, Bengio and Courville that can be purchased on \href{https://www.amazon.com/Deep-Learning-Adaptive-Computation-Machine/dp/0262035618}{Amazon} and read for free at \url{http://www.deeplearningbook.org/}.

\paragraph{Popular Books:} We will also be reading the non-fiction novel \qu{The Signal and the Noise} by Nate Silver which can also be purchased on \href{https://www.amazon.com/Signal-Noise-Many-Predictions-Fail-but/dp/0143125087}{Amazon}. This is \textit{required} --- you will have homework questions directly from this book. We will also be reading \qu{Preditive Analytics, Data Mining and Big Data} by Steven Finlay that can be purchased on \href{https://www.amazon.com/Predictive-Analytics-Data-Mining-Misconceptions/dp/1349478687}{Amazon} and it is also available online from the \href{https://link-springer-com.queens.ezproxy.cuny.edu/book/10.1057%2F9781137379283}{Queens College library system}. 

\paragraph{Computer Software:} You need your own personal computer, laptop preferred. We will be using \texttt{R} which is a free, open source statistical programming language and console available for all operating systems. Please download the latest version from: \url{http://cran.mirrors.hoobly.com/}. You will be expected to do programming. I recommend the IDE \texttt{RStudio} available for free at \url{https://www.rstudio.com/products/rstudio/download/}.

\paragraph{Source Control:} You will be expected to use \texttt{git} and have a \url{github.com} account with a repository named \texttt{QC\_MATH\_342}. You will use this repository to submit coding homework assignments (and theory assignments if you use \LaTeX).


\paragraph{Book on \texttt{R}:} We will be making some use of \qu{R for Data Science} by Wickham and Grolemund which can be purchased on \href{https://www.amazon.com/R-Data-Science-Hadley-Wickham/dp/1491910399}{Amazon} or read online at \url{http://r4ds.had.co.nz/}.


\input{../../syllabi/_the650section}

\input{../../syllabi/_announcements_on_slack}

\input{../../syllabi/_use_of_slack}

\section*{Class Meetings}

There are 42 scheduled meetings. Of these, 26 will be lectures, 10 will be labs, 2 will be midterm exams (see the schedule on page~\pageref{subsec:exam_schedule}) which are in class and 2 will be review periods before the exams, 1 will be a review lab and 1 will be help with the final paper. I am \inred{canceling} Monday, May 17 (the last meeting) due to a Jewish holiday. This meeting would have been for help on the final paper. We will decide when to have this session during finals week that fits the majority's schedule.

\subsection*{Lectures}

Lectures will be on zoom and will be split usually into two periods: theory and practice. The first is a standard \qu{chalkboard} lecture where we learn concepts and the second will be using the \qu{computer/projector} to see the concepts in action in the \texttt{R} language. %I have a no computer / tablet / phone policy during the theory component of the lectures (only pen / pencil and paper) but you are highly recommended to have the laptop during the second part.

\input{../../syllabi/_zoom_policies}


\subsection*{Lecture Schedule}

Below is a tentative schedule of the theory and practice topics covered by lecture number. 

\begin{enumerate}[(1)]
\item \textbf{Theory:} Review of syllabus, introducing science and modeling, definition of phenomena / the response $y$, reality vs. approximation, measurement vs. prediction and simulation, definition of learning from data, model validation, heuristics, ambiguous models, mathematical models, causal inputs, response spaces $\mathcal{Y}$, definition of features $\x$ and its feature space $\mathcal{X}$.

\textbf{Practice:} Short history of \texttt{R}, introduction to RStudio, arithmetic, assignment of variables, mathematical functions, logical operations, numeric / integer / boolean data types, vectors, sequences, subsetting, sorting, taxonomy of illegal values.

\item \textbf{Theory:} Feature spaces, binary, categorical and continuous data types, definition of metrics, ordinal codings, sample size $n$ vs. number of features $p$, error due to ignorance of information $\delta$, optimal response surface $f$, definition of training data $\mathbb{D}$, definition of candidate function set $\mathcal{H}$, definition of prediction function $g$, algorithms that produce prediction functions $\mathcal{A}$, definition of misspecification error, definition of optimal candidate function $h^*$, irreducible error $\mathcal{E}$, estimation error and residuals $e$.

\textbf{Practice:} Realizations of popular random variables, PDF / PMF / CDF / empirical CDF computations, quantile computations, factor-type variables, matrix data type and its critical functions, if / if-else / else / switch programmatic control, for / while / repeat loops, console printing, errors and warnings, try-catch control.

\item \textbf{Theory:} Visualization of training data $\mathbb{D}$, threshold models for classification, null model for classification $g_0$, concept of a parameter $\theta$ and parameter space $\Theta$, degrees of freedom, definition of objective function / error function, accuracy, sum of squared error (SSE), optimization within $\mathcal{A}$, linear threshold models, perception learning algorithm (PLA), introduction to neural networks.

\textbf{Practice:} Review of hashing and the list data type, the array data type for general tensors, naming for vectors / matrices / tensors, introduction to specifying functions, arguments, argument defaults, creating data matrices, tabling multiple features, the dataframe data type.

\item \textbf{Theory:} Review of lines in multiple dimensions with Hesse Normal Form, derivation of the support vector machine (SVM) using maximum margin objective, hinge error, SVM using the Vapnik objective, defintion of hyperparameters $\lambda$, $K$-nearest neighbors algorithm (KNN).

\textbf{Practice:} Installing and loading libraries from CRAN and githuv, public vs private functions and scoping, loading datasets from libraries / files / URLs, creating threshold models, writing the PLA, matrix operations: arithmetic / transpose / inverse / rank / trace, optimization algorithms e.g. Nelder-Mead, writing the KNN algorithm, using the SVM library and setting the hyperparameter.

\item \textbf{Theory:} Null model for regression $g_0$, linear models for continuous responses $\bbeta$, minimization of SSE for $p=1$ using basic calculus to arrive at the ordinary least squares (OLS) solution $\b$, review of covariance of two random variables $\sigma_{X,Y}$ and its estimate $s_{X,Y}$, review of correlation $\rho_{X,Y}$ and its estimate $r_{X,Y}$.

\textbf{Practice:} Computing sample covariance and correlation in the context of the OLS algorithm for $p=1$, computing OLS error metrics, the formula object, using the \texttt{lm} function, visualizing the OLS line atop a scatterplot, computing predictions in OLS, OLS in the boston housing data

\item \textbf{Theory:} Error metrics for regression: SSE, mean squared error (MSE) and root mean squared error (RMSE), approximate prediction confidence intervals using RMSE, sum of squares total (SST), concept of proportion of variance explained $R^2$, class of models with $R^2 < 0$, perfect fit models with $R^2 = 1$.

\textbf{Practice:} OLS on the Galton height data and the etymology of the word \qu{regression} in statistics, computing OLS in the case of categorical variables (ANOVA), dummifying variables, computing model matrices.

\item \textbf{Theory:} OLS estimates being the group averages with one binary feature, independence, dependence, association, correlation, OLS estimates with $p>1$, design matrix $X$, vector derivative properties: constant scalars, multiples, multiplication, quadratic forms, general OLS solution, review of matrix inverses, transposes, rank and symmetric matrices.

\textbf{Practice:} Visualizing $R^2$ for a model using error density estimation, computing general OLS estimates in multiple dimensions from scratch, making predictions using the \texttt{predict} interface for modeling.

\item \textbf{Theory:} OLS predictions as linear transformations, review of the linear algebra concepts of dimension, length, norm, subspace, linear in/dependence, column space, derivation of the orthogonal projection matrix for one dimension via law of cosines, outer products, idempotency.

\textbf{Practice:} Eigendecomposition, computing error metrics for general OLS, computing the null model.

\item \textbf{Theory:} Derivation of the orthogonal projection in multiple dimensions, equivalence of the OLS algorithm with orthongal projection, definition of the hat matrix $H$, review of the linear algebra concepts of eigenvectors and eigenvalues, computing the eigenvectors and eigenvectors of $H$.

\textbf{Practice:} Computing the hat matrix $H$ for the null model and in general, confirming its eigendecomposition and idempotency and rank, using $H$ to find the OLS predictions and residuals and verifying their orthogonality.

\item \textbf{Theory:} Verification of the symmetry and idempotency of $H$, proving that one multidimensional orthogonal projection is in general not the same as the sum of orthogonal projections in the component dimensions except if the component dimensions themselves are orthogonal, review of orthonormal matrices $Q$, proving the equivalence of $H = QQ^\top$, definition of $X = QR$ decomposition, the Gram-Schmidt algorithm, computing $R$, deriving the least squares estimate using $Q$ and $R$.

\textbf{Practice:} Computing $QR$ decomposition and confirming the orthonormality of $Q$, verifying the OLS predictions are the same with $Q$, writing the Gram-Schmidt algorithm, an overview of the piping / chaining concept in modern programming and in \texttt{R} with demos. 

\item \textbf{Theory:} Definition of sum of squares for the regression $SSR$, proving the sum of squares identity $SST = SSR + SSE$, showing that if $p$ increases by one dimension, then SSR is obligated to increase forcing $R^2$ higher, definition of overfitting in modeling, definition of chance capitalization, demonstrating that full overfitting in OLS leads to $H = I$, a superficial introduction to regularization (lasso regression and ridge regression).

\textbf{Practice:} Demo showing the iterative addition of a feature and $R^2$ monotonically increasing and RMSE monotonically decreasing (overfitting), demonstrating that random vectors are never truly orthogonal and thus their projections are non-zero, a lasso fit and a ridge fit of a dataset with $p \geq n$.

\item \textbf{Theory:} Definition of in-sample error metrics vs. out-of-sample (oos) error metrics, splitting $\mathbb{D}$ into $\mathbb{D}_{\text{train}}$ and $\mathbb{D}_{\text{test}}$ via split constant $K$, definition of \qu{honest} validation via oos error metrics, definition of the final model $g_{\text{final}}$ definition of underfitting, tracing the underfitting-overfitting complexity curve, definition of optimal-complexity models.

\textbf{Practice:} A more full demo of overfitting with visualizations, demo of consistency of OLS estimates, code to create train-test splits, demonstration that oos error is larger than in-sample error in the scenario of the model being overfit.

\item \textbf{Theory:} Definition of raw features versus derives features, increasing complexity in $\mathcal{H}$ using polynomial functions of raw features, interpretation of OLS coefficients $\b$, Weierstrauss Approximation Theorem, OLS with polynomial features, definition of the full rank Vandermonde matrix, definitions of interpolation vs. extrapolation.

\textbf{Practice:} Square, cube and higher order polynomial fitting both raw and orthogonal with visualization, overfitting with high-degree polynomials, demonstration of extrapolation in models of many different polynomial degrees, prediction with polynomial models, extrapolation in the Galton height data.

\item \textbf{Theory:} OLS using the log transformation on both features and response and interpretation of $\b$, derivation the log change is approximately percentage change, definition of first-order interactions in OLS, interpretations of coefficients in interaction models.

\textbf{Practice:} Log-linear model fitting and log-log linear model fitting, logging response to reduce the effect of influential observations, the grammar of graphics and the \texttt{ggplot} package to create histograms, scatterplots, box-whisker, violin plots, smoothing plots, overloading plots with many features, faceting, coloring, aesthetics and themes, using color illustrations and faceting to visualize potential first-order interactions in a linear model, fitting interaction models.

\item \textbf{Theory:} oos error metrics as estimates of model generalization error, sources of variance in these estimates, mitigation by adjusting $K$, further mitigation by using cross-validation (CV), $K$-fold CV and its aggregated error estimates, approximation confidence intervals for generalization error, discussion of reasonable values of $K$ in practice.

\textbf{Practice:} Simulating many different train-test splits to underscore that $K$ trades bias vs. variance in the generalization estimate, writing code for $K$-fold CV, using the package \texttt{mlr3} to automate $K$-fold CV.

\item \textbf{Theory:} Introduction of the fundamental problem of \qu{model selection} of candidates $g_1, \ldots, g_M$, model selection with honest validation via splitting $\mathbb{D}$ into $\mathbb{D}_{\text{train}}$, $\mathbb{D}_{\text{select}}$ and $\mathbb{D}_{\text{test}}$ via split constants $K_{\text{select}}$ and $K_{\text{test}}$, procedure to select best model among $M$ candidates and validation of the best model.

\textbf{Practice:} Writing code for the model selection procedure, review of basic \texttt{C++}, optimizing \texttt{R} code via the \texttt{Rcpp} package, benchmarking routines that require heavy looping between \texttt{Rcpp} and base \texttt{R}, benchmarking routines that require heavy recursion between \texttt{Rcpp} and base \texttt{R}.

\item \textbf{Theory:} Double-CV in the model selection procedure using inner folds and outer folds, discussion of reasonable values of $K_{\text{select}}$ and $K_{\text{test}}$ in practice, applying the model selection procedure to grid searching to locate the best value of hyperparameters $\lambda$ in algorithms that require $\lambda$, definition of stepwise modeling using the model selection procedure via the underfitting-overfitting complexity curve concept, stepwise OLS with a large basis of candidate terms. 

\textbf{Practice:} Using the package \texttt{mlr3} to automate the double-CV using inner and outer loops, using the package \texttt{mlr3} to automate the locating of optimal hyperparameters, demo of forward stepwise linear modeling and tracing the underfitting-overfitting complexity curve.

\item \textbf{Theory:} Definition of hyperrectangle basis for $\mathcal{X}$ and its OLS solution, unfeasibility of this algorithm in high $p$, introduction of the regression tree algorithm.

\textbf{Practice:} Binning model demo and visualization for varying bin sizes, introduction to data wrangling using the packages \texttt{dplyr} and \texttt{data.table}: filtering, sorting, grouping, summarizing, feature derivation, dataframe joining (left, right, inner, full, between / overlap), benchmarking the two libraries.

\item \textbf{Theory:} Full specification of regression tree algorithm: definition of a binary tree, definition of orthogonal-to-axes splits, nodes vs. leaves, left-right SSE weighting, leaf assignments, overfitting and tree-pruning.

\textbf{Practice:} Using the \texttt{YARF} package to produce regression trees, querying tree stats, visualizing trees and tree model predictions, tree differences by the pruning hyperparameter.

\item \textbf{Theory:} MSE of $g$ decomposition into bias and irreducible error for one $\mathbb{D}$, MSE of $g$ decomposition into bias, irreducible error and variance for multiple $\mathbb{D}$'s, MSE decomposition of $M$ different $g$'s averaged, strategies to eliminate bias and variance, non-parametric bootstrap sampling, Breiman's concept of bootstrap aggregation (bagging), correlation $\rho$ among the bootstrapped models, out-of-bag (oob) observations, validation in bagging via oob samples.

\textbf{Practice:} Visualizing $M$ bagged trees, demonstrating near zero bias, demonstrating variance reduction as $M \rightarrow \infty$, a comparison to OLS and high degree polynomial models, demonstration of generalization error improvement, demonstration of validation in bagging.

\item \textbf{Theory:} Demonstrating the bias for regression trees is near zero, reducing correlation among the bootstrapped models using feature sampling, introduction of random forests (RF) algorithm.

\textbf{Practice:} Demonstration that RF decreases $\rho$ and demonstration that it outperforms bagging in both regression and classification.

\item \textbf{Theory:} A basic discussion of \qu{causality} from a philosophical perspective, directed causal graphs, correlation vs. causation, incidental effects, lurking variables, spurious correlation, causation is defined by manipulation, a quick definition of randomized experimentation, real-world causal diagrams, wrong interpretations of $\b$ in OLS, the highly limited but true / complete paragraph-long interpretation of $\b$ in OLS, a discussion of how OLS regression accomplishes estimation of single features with other features \emph{ceteris paribus}.

\textbf{Practice:} Demos of whimsical spurious correlations, demonstration that spurious correlations are easy to find in simulation, a nice illustration of correlation without causation using an OLS model that reveals the true interpretation of $\b$.

\item \textbf{Theory:} Introduction of classification tree algorithm, the gini metric, leaf assignments, the two errors: false negatives and false positives, the 2$\times$2 confusion matrix and its metrics: precision, recall, accuracy, $F_1$ metric, false discovery rate, false omission rate.

\textbf{Practice:} Using the \texttt{YARF} package to produce classification trees, querying tree stats, visualizing trees and tree model predictions, tree differences by the pruning hyperparameter, measuring the two errors, computing confusion matrices and the other metrics.

\item \textbf{Theory:} Missing data mechanisms (MDMs): missing completely at random, missing at random, not missing at random with examples, strategies to handle missingness: listwise deletion, imputation, multiple imputation, miss forests algorithm and its convergence, the concept of \qu{retaining} missingness even after imputation, introduction of probability estimation using the independent bernoulli random variable model, optimal probability function, likelihood of $\mathbb{D}$, $\mathcal{H}$ for probability functions, generalized linear modeling, link functions: logistic / probit / complementary log-log, numerical approximations to the likelihood optimization.

\textbf{Practice:} An example of a dataset with missingness, assessing the different MDMs, listwise deletion, imputation and the \texttt{missForest} package for the recommended imputation, creating the missingness dummies as derives features in $X$.

\item \textbf{Theory:} Definition of logistic regression (LR), log-odds interpretation of LR $\b$, prediction in LR, error metrics for probability estimation: Brier and Log scoring rules, classification modeling from probability regression, optimal asymmetric cost modeling, response-operator curves (ROC), detection-error tradeoff curves (DET).

\textbf{Practice:} Fitting LR models using the \texttt{glm} package, predicting with LR models, validating creating asymmetric cost classifiers in LR models, constructing ROC and DET plots using LR models, locating optimal models with minimal cost. 

\item \textbf{Theory:} Causal diagrams, confounding, correlation does not imply causation, correct interpretation of OLS estimates, asymmetric cost classification in tree models, introduction to boosting as a meta-algorithm, adaboost

\textbf{Practice:} demonstrating confounding and linear models by hiding the confounder and then revealing it, demonstration of asymmetric classification using trees, demonstration of boosting using package \texttt{xgboost}.
\end{enumerate}



\input{../../syllabi/_lecture_upload}


\subsection*{Labs}

Labs  will be on zoom during the Thursday morning meetings. Sometimes we will spend some of the two hours doing a practice lecture but the majority of this time will be your time. You will take turns \qu{driving} the coding in front of the class, working on exercises that you will finish for homework. Thus we will spend most a lot of time talking through problem solving skills in data science.

\section*{Homework}

Homework will be split into \textit{theory} and \textit{practice} (called \qu{labs}). This course will be the \qu{writing in the major course} next year. Thus, a portion of each theory and practice homework will involve writing \textit{English} and being graded on \textit{English}.

\subsection*{Theory Homework}

\input{../../syllabi/_theory_hws_text}
\input{../../syllabi/_theory_hws_submission_text}



\subsection*{Practice Homework (Labs)}

These will almost exclusively consist of short and medium coding exercises in \texttt{R}. Most of the assignment will be done for you and your peers during the Friday lab session.


\input{../../syllabi/_philosophy_hws}

\input{../../syllabi/_time_spent_hws}

\input{../../syllabi/_late_hw_policy}

\input{../../syllabi/_latex_hw_bonus_policy}

\input{../../syllabi/_hw_ec_policy}

\input{../../syllabi/_hw_0}

\section*{Writing Assignments}

There will be two writing assingments. (1) A \qu{philosophy of modeling} essay. Here you will coalesce the non-mathematical material that is crucial to this class. The purpose is to make you truly understand the modeling process and its limitations from start to finish. (2) A final project. Here you will use build a predictive model using a dataset. This is the capstone project for the entire data science and statistics major and it is where you will tie everything together.

This class will soon be the writing in the major course. Thus, writing is a major part of the curriculum herein.

\section*{Examinations}

\input{../../syllabi/_examination_text}

Since the is the capstone course, there is no final exam, but a large final project. There will be two midterm exams and the schedule is below.

\subsection*{Exam and Major Assignment Schedule}\label{subsec:exam_schedule}

\begin{itemize}
\itemsep -0.0em 
\item Midterm examination I will be Thurs, March 24 in class with the first review session on the Wednesday prior
\item The philosophy of modeling paper's first draft is due Sunday, Mar 21 at 11:59PM
\item Midterm examination II will be Thurs, May 13 in class with a review on the Wednesday prior
\item The final project is due Sunday, May 23 11:59PM
\end{itemize}

\subsection*{Exam Policies and Materials}

\input{../../syllabi/_examination_policies}


I also allow \qu{cheat sheets} on examinations. For both midterms, you are allowed to bring \ingreen{two} 8.5'' $\times$ 11'' sheet of paper (front and back). \inred{Four sheets single sided are not allowed.} On this paper you can write anything you would like which you believe will help you on the exam. %For the final, you are allowed to bring three 8.5'' $\times$ 11'' sheet of paper (front and back). \inred{Six sheets single sided are not allowed.} I will be handing back the cheat sheets so you can reuse your midterm cheat sheets for the final if you wish. 




\input{../../syllabi/_cheating_on_exams_and_missing_exams}
\input{../../syllabi/_special_services}

\input{../../syllabi/_class_participation}

\input{../../syllabi/_zoom_attendance}


\section*{Grading and Grading Policy}\label{sec:grading}

Your course grade will be calculated based on the percentages as follows: 

\begin{table}[h]
\centering
\begin{tabular}{l|l}
Theory Homework & 9\% \\
Labs & 14\% \\
Midterm Examination I & 18\%\\
Midterm Examination II* & 18\%\\
Philosophy of Modeling Paper & 9\% \\
Final Project with Writeup & 22\% \\
Class participation & 5\% \\
Attendance & 5\% 
\end{tabular}
\end{table}
\FloatBarrier

\noindent *The second midterm is not cumulative. It only covers material \textit{after} midterm I.


\input{../../syllabi/_advanced_course_grade_distribution}

\input{../../syllabi/_grade_checking_on_gradesly}

\input{../../syllabi/_auditing_policy}




\end{document}
